\documentclass[a4paper]{article}

%% Language and font encodings
\usepackage[english]{babel}
\usepackage[utf8x]{inputenc}
\usepackage[T1]{fontenc}

%% Sets page size and margins
\usepackage[a4paper,top=3cm,bottom=2cm,left=2cm,right=2cm,marginparwidth=1.75cm]{geometry}

%% Useful packages
\usepackage{amsmath}
\usepackage{graphicx}
\usepackage{array}
\usepackage[colorinlistoftodos]{todonotes}
\usepackage[colorlinks=true, allcolors=blue]{hyperref}
\usepackage{titlesec}

\usepackage{amsmath}

\titlespacing*{\section}
{0pt}{6.5ex plus 1ex minus .2ex}{3.3ex plus .2ex}
\titlespacing*{\subsection}
{0pt}{5ex plus 1ex minus .2ex}{2.5ex plus .2ex}


%===========================================================================
%% Front
%===========================================================================

\title{Proposal for an application}
\author{Lisa Cattalani and Andrea Pari \\
Alma Mater Studiorum - University of Bologna, via Venezia 52 \\
47023 Cesena, Italy \\
\textit{lisa.cattalani@studio.unibo.it, andrea.pari6@studio.unibo.it}}

%===========================================================================


\begin{document}
\maketitle
\newpage 
\tableofcontents
\newpage



%===========================================================================
\section{Introduction}
%===========================================================================

The current report describes the entire software development process adopted to analyze, design and implement a specific software system. The process has been divided into two phases: the first phase concerns the definition of a model of the software system, while the second phase concerns the implementation of the software system carried out after that the product owner has viewed the model.



%===========================================================================
\section{Vision}
%===========================================================================

We want to show how to manage the development of a distributed, heterogeneous and embedded software system by focusing on the analysis and design of the system. The software system at issue will be designed and implemented as an application in the field of \textit{Internet of Things}.



%===========================================================================
\section{Goals}
%===========================================================================

The goal is to build a software system able to evolve from an initial prototype (defined as result of a problem analysis phase) to a final and testable product. This goal will be achieved by working in a team and by "mixing" in a proper (pragmatically useful) way an agile (SCRUM) software development with modeling.



%===========================================================================
\section{Requirements }
%===========================================================================

%===========================================================================
\subsection {The context}
%===========================================================================

A differential drive robot (called from now on robot) must reach an area (B) starting from a given point A. To reach the area B, the robot must cross an area equipped with N (N $\geq$ 1) distance sensors (sonars). The signal emitted by each sonar is reflected by a wall put in front of it at a distance of approximately 90 cm.
Moreover:
\begin{itemize}
	\item The section of the wall in front of each sonar is painted with a different illustration.
	\item The robot is equipped with a distance sensor (sonar) and (optionally) with a Web Cam both positioned in its front. It owns also a Led.
	\item The robot should move from A to B by travelling along a straight line, at a distance of approximately 40-50 cm from the base-line of the sonars.
\end{itemize}


%===========================================================================
\subsection {The work to do}
%===========================================================================

Design and build a (prototype of a) software system such that:
\begin{itemize}
	\item Shows the sonar data on the GUI associated to a console running on a conventional PC.
	\item Evaluates the expression
	
	(s\textsubscript{k} + s\textsubscript{k+1} + ... s\textsubscript{N}) / (N - k + 1)
	
	where k is the number of the first sensor not reached by the robot and s\textsubscript{k} is the value of the distance currently measured by that sensor. If the value of the expression is less than a prefixed value DMIN (e.g. DMIN = 70), play an alarm sound.
	\item When the robot reaches the area in front of a sonar, it:
	\begin{itemize}
		\item first stops;
		\item then rotates to its left of approximately 90 degrees;
		\item starts blinking a led put on the robot;
		\item takes a photo of the wall (in a simulated way only, if no WebCam is available) and sends the photo to console by using the MQTT protocol;
		\item rotates to its right of approximately 90 degrees to compensate the previous rotation;
		\item stops the blinking of the led and continues its movement towards the area B;
	\end{itemize}
	\item When the robot leaves the area in front to the last sonar, it continues until it arrives at the area B.
	\item Stops the robot movement as soon as possible:
	\begin{itemize}
		\item when an obstacle is detected by the sonar in front of the robot;
		\item when an alarm sound is played;
		\item the user sends to the robot a proper command (e.g. STOP).
	\end{itemize}
	\item Makes it possible to restart the system (by manually repositioning the robot at point A) without restarting the software.
\end{itemize}



%===========================================================================
\section{Requirements analysis }
%===========================================================================

The first step to analyze correctly the set of requirements is to read the requirements text in order to understand the applicative domain.\newline\newline
After a careful reading, we have considered appropriate to do a short interview with the product owner in order to clear up any doubt come to light during the requirements text reading. The interview is reported below.\newline\newline
\textbf{Q: How the robot starts its movement when placed on the given point A?}\newline
A: The user must send a START command from the console.\newline\newline
\textbf{Q: Which must be the robot behaviour when it reaches the area B?}\newline
A: It must stops his movement, perhaps because an obstacle is in front of it.\newline\newline
\textbf{Q: Which sonars are responsible to detect the obstacles along the path of the robot?}\newline
A: The sonars along the path of the robot can detect any type of object in front of them (the robot in particular). The sonar on the robot can detect obstacles in front of it, too.\newline\newline
Then, we have proceeded to create a glossary of terms in order to avoid any ambiguity or misunderstanding during the next analysis phases. The terms have been reported in \textbf{[Tab. 1]}. For each term, the related definition and eventual synonims are shown. \hfill \break

\def\arraystretch{1.8}
\begin{tabular}{ | m{3cm} | m{9cm} | m{3cm} | }

    \hline
    
	\textbf{Term} & \textbf{Definition} & \textbf{Synonyms}\\ 
	
	\hline
	
	\textit{Prototype} & A prototype is an incomplete version of a software system used to demonstrate concepts, to try some project options and to find out problems and possible solutions. A prototype can present all or a few characteristics of the final system. & Model\\
	
	\hline
	
	\textit{Differential drive robot} & A differential drive robot is a two-wheeled drive robot system with independent actuators for each wheel. The name refers to the fact that the motion vector of the robot is the sum of the independent wheel motions. & Robot\\
	
	\hline
	
	\textit{Distance sensor} & A distance sensor is a device able to detect the presence of an object (i.e. an obstacle) in front of itself. & Sonar\newline Sensor\\
		
	\hline
	
	\textit{Console} & In the current domain, the console represents the means by which the user and the robot interact with each other. Through the console, the user sends commands to the robot and receives the data of the photos taken by it. & Command panel\\
	
	\hline
	
	\textit{GUI} & This term refers to the \textit{Graphical User Interface} associated to the console. Through this interface, the user can view the data sent to the console by the sonar of the robot. & Interface\newline User interface\\
	
	\hline
	
	\textit{Alarm sound} & The sound emitted by the console when the value calculated through the expression of the distance to the next sonar is under the prefixed threshold. & Alarm\\
	
	\hline
	
	\textit{MQTT protocol} & \textit{MQTT (MQ Telemetry Transport)} is a communication protocol based on the publish/subscribe pattern. MQTT represents the means by which the robot sends to the console the data of the photos. &\\
	
	\hline
	
	\textit{Obstacle} & Any type of object located along the path of the robot that can potentially prevent it to reach the area B. & \\
	
	\hline
	
	\textit{User} & The user is the one who will use the software system, giving commands to the robot through the console and receiving the data of the photos taken by it. & End user\\
	
	\hline
	
	\textit{Command} & A command is an order given by the user to the robot using the console on the PC. &\\
	
	\hline
	
	\textit{Restart the system} & The process through which the robot is manually repositioned on the given starting point A without the need to restart the software. & Restart\newline Reboot \newline Reboot the system\\
	
	\hline 

\end{tabular}

\begin{center}
\textit{[ Tab. 1 - Glossary ]}
\end{center}


%===========================================================================
\subsection {User Stories}
%===========================================================================

A \textbf{user story} is an informal description of one or more features required by a software system in order to satisfy a need of the stakeholders. They are expressed in natural language, in order to be understandable by anyone.\newline\newline
A user story is then described from the end user's point of view, and not from the system's point of view. This means that the description is made without expressing any technical details about the analysis, the design and the implementation of the software system.\newline\newline
Each user story will be defined using the following format:
 \hfill \break

\def\arraystretch{1.8}
\begin{table}[h]
\centering
\begin{tabular}{ | m{8.5cm} | }

    \hline
	
	\textit{"As a \textbf{user}, I want <GOAL> [so that <BENEFIT>]"}\\
	
	\hline 

\end{tabular}
\end{table}

The stories collected are shown below.

\begin{enumerate}

	\item As a \textbf{user}, \textbf{I want} to use a console running on a PC \textbf{so that} I can interact with the robot and send it specific commands to start and stop its movements.
	\item As a \textbf{user}, \textbf{I want} that both the sonar on the robot and the sonars along the path can detect objects {so that} collisions with them can be avoided.
	\item As a \textbf{user}, \textbf{I want} that an alarm sound was played when the value calculated by the expression of the distance is under a prefixed threshold {so that} I can be alerted.
	\item As a \textbf{user}, \textbf{I want} that the robot stops its movement whenever it reaches the area in front of each sonar along the path. Moreover, {I want} that the robot rotates to its left \textbf{so that} it can take a photo of the illustration painted on the wall and send it to the console.
	\item As a \textbf{user}, \textbf{I want} a Graphical User Interface associated with the console {so that} I can see the data sent by the robot, including the photos.
	\item As a \textbf{user}, \textbf{I want} that the robot stops its movements when it reaches the area B.
	\item As a \textbf{user}, \textbf{I want} to restart the system by manually repositioning the robot at the given point A (i.e. without restarting the software).
	
\end{enumerate}


%===========================================================================
\subsection {Sum up of functional requirements}
%===========================================================================

Now that the applicative domain has become more clear and definite, we proceed to list in broad terms the requirements that the final system will must possess in order to satisfy the needs of the product owner. The three groups identify the main macro-components of the system.

\begin{enumerate}

	\item \textbf{Robot}
		\begin{enumerate}
			\item Movements
				\begin{enumerate}
					\item Move forward
					\item Rotate on its left
					\item Stop
				\end{enumerate}
			\item Detect obstacles
			\item Blink a led
			\item Photos
				\begin{enumerate}
					\item Take photos
					\item Send photos to the console
				\end{enumerate}
		\end{enumerate}
	\item \textbf{Console and GUI}
		\begin{enumerate}
			\item Send commands to the robot
			\item Evaluate distance expression
			\item Play an alarm sound
			\item Show sonar data
		\end{enumerate}
	\item \textbf{Sonars along the path}
		\begin{enumerate}
			\item Detect obstacles
			\item Send distance data to the console
		\end{enumerate}
	
\end{enumerate}


%===========================================================================
\subsection {User interaction and scenarios}
%===========================================================================

In order to illustrate in a comprehensive and unambiguous manner the functional interaction between the (external) users and the system, we will use the formalism given by the UML meta-model. In \textbf{[Fig. 1]} an use case diagram is shown in order to describe the interaction between an user and the system through a set of "actions", which represent the use case of the system.\newline\newline
Note that only what an user can directly do with the system is shown.
\newline

\begin{center}
	[QUI IL DIAGRAMMA]
\end{center}

\begin{center}
	\textit{[ Fig. 1 - Use case diagram ]}
\end{center}

\hfill \break

Now we will describe the scenarios according to the described use cases. \hfill \break

\def\arraystretch{1.8}
\begin{tabular}{ | m{4cm} | m{11cm}| }
	
	\hline
	
	\textbf{ID / Name}&\textbf{\#1 - Robot (re-)positioning}\\ 
	
	\hline
	
	\textbf{Description}&Positioning the robot on the given point A.\\

	\hline
	
	\textbf{Actor}&User\\
	
	\hline	

	\textbf{Pre-conditions}&The user must have the robot available.\\

	\hline

	\textbf{Main scenario}&The user places the robot on the given point A in such a way that:
	\begin{enumerate}
		\item the robot can move straight towards the area B;
		\item the distance of the robot from the base-line sonars along the path was of approximately 40-50 cm.
	\end{enumerate}
	Then, the robot can move.\\
	
	\hline	
	
	\textbf{Alternative scenarios}&---\\
	
	\hline
	
	\textbf{Post-conditions}&The robot must reboot itself every time it is placed on the point A.\\

	\hline

\end{tabular}

\begin{center}
	\textit{[ Tab. 2 - Scenario 1 ]}
\end{center}

\hfill \break

\def\arraystretch{1.8}
\begin{tabular}{ | m{4cm} | m{11cm}| }
	
	\hline
	
	\textbf{ID / Name}&\textbf{\#2 - Sending commands}\\ 
	
	\hline
	
	\textbf{Description}&Sending commands to the robot through a console running on a conventional PC.\\

	\hline
	
	\textbf{Actor}&User\\
	
	\hline	

	\textbf{Pre-conditions}&
	\begin{enumerate}
		\item The console is running on the PC.
		\item The system is ready to accept commands from the console.
	\end{enumerate}\\

	\hline

	\textbf{Main scenario}&The user interacts with the robot in the following way:
	\begin{enumerate}
		\item the user sends a command to the robot through the console on the PC;
		\item the system receives the command, evaluates it and executes it.
	\end{enumerate}\\
	
	\hline	
	
	\textbf{Alternative scenarios}&---\\
	
	\hline
	
	\textbf{Post-conditions}&The robot must:
	\begin{enumerate}
		\item move forward if the system receives a START command;
		\item stop itself immediately if the system receives a STOP command.
	\end{enumerate}\\
	
	\hline
	
\end{tabular}

\begin{center}
	\textit{[ Tab. 3 - Scenario 2 ]}
\end{center}

\hfill \break

\def\arraystretch{1.8}
\begin{tabular}{ | m{4cm} | m{11cm}| }
	
	\hline
	
	\textbf{ID / Name}&\textbf{\#3 - Displaying data}\\ 
	
	\hline
	
	\textbf{Description}&Viewing the data on the GUI associated to the console.\\

	\hline
	
	\textbf{Actor}&User\\
	
	\hline	

	\textbf{Pre-conditions}&The console is running on the PC.\\

	\hline

	\textbf{Main scenario}&The user sees and checks on the GUI the data sent by the robot, including the photos taken by it.\\
	
	\hline	
	
	\textbf{Alternative scenarios}&---\\
	
	\hline
	
	\textbf{Post-conditions}&---\\
	
	\hline
	
\end{tabular}

\begin{center}
	\textit{[ Tab. 4 - Scenario 3 ]}
\end{center}

\hfill \break


%===========================================================================
\subsection {(Domain) Model}
%===========================================================================

We will now describe the (domain) model of the system according to the requirements expressed by the product owner. So, for each entity of the system, the model will describe them on the three dimensions of structure, interaction and behaviour.\newline\newline
{\large\textbf{Robot}}\newline\newline
\underline{Structure.} The robot is a non-atomic reactive entity. It is composed by the following parts.
\begin{enumerate}
	\item \textit{Wheels:} circular tire that is intended to rotate on an axle bearing.
	\item \textit{Eletric motors:} electrical machines able to move and rotate the wheels. There is a motor for each wheel.
	\item \textit{Sensor:} electronic component able to detect events or changes in its environment and send the information to other electronics. In particular, the sensor must be able to detect objects.
	\item \textit{H-bridge:} electronic circuit that allows the motors to run forward.
	\item \textit{Led:} passive output device that emits a blinking light when activated.
	\item \textit{Raspberry Pi:} computational hardware unit able to control and handle motors, sensors, led and generally any I/O operation with other computational entities.
	\item \textit{Frame/Platform:} plan in which all the components described above are placed and installed.
	\item \textit{Powerbank:} portable device that can supply power from its built-in batteries.
	\item \textit{WiFi USB:} device that allows Raspberry to connect to the network and communicate remotely with other computational entities.
	\item \textit{Various hardware accessories:} wires and other minor hardware components.
\end{enumerate}
\underline{Interaction.} The robot interacts only with the console, sending it the photos taken by it.\newline\newline
\underline{Behaviour.} The general behaviour of the robot can be summed up as follows:
	\begin{enumerate}
		\item to move along a linear path starting from a given point A and moving towards an area B (standard behaviour);
		\item to react to particular events that can change its standard behaviour;
		\item to receive specific commands able to start and stop its movements.
	\end{enumerate}
So, generally speaking, its behaviour follows both an event-driven model and a message-based model.
\newline\newline\newline
{\large\textbf{Console and GUI}}\newline\newline
\underline{Structure.} The console is a reactive computational entity running on a PC connected to the system through the network. He is associated with a GUI.\newline\newline
\underline{Interaction.} The console interacts with the robot and with the N sonars placed along the path of the robot. In particular, the console receives the data of the photos taken by the robot and gets the data emitted by the sonar in order to calculate the value of the distance expression. Moreover, the console interacts with the robot by sending it commands to start and stop its movements.\newline\newline
\underline{Behaviour.} The console simply display through the GUI associated with it the data acquired and sent by the robot. Moreover, it plays an alarm sound if the value of the expression calculated by it is less than the prefixed threshold. So, its behaviour is reactive with respect to the data sent by the sonar and active in the way it evaluates them.
\newline\newline\newline
{\large\textbf{Sonar subsystem}}\newline\newline
\underline{Structure.} With the term \textit{sonar subsystem} we refer to the set of sonars placed along the path crossed by the robot. The sonar subsystem is obviously a composed entity consisting of N sonars. Each sonar is an independent input entity connected to a Raspberry.\newline\newline
\underline{Interaction.} Each sonar in the sonar subsystem interacts with the console by sending to it the value of the distance from an eventual object placed in the sonar area.
\newline\newline
\underline{Behaviour.} The sonar subsystem has an active behaviour. Its job is to sense the environment around it in order to detect eventual object in its area, to get the distance from it and to send the data to the console.


%===========================================================================
\subsection {Test plan}
%===========================================================================

We conclude the phase of requirement analysis by reporting a first plan of tests that will be performed in a simulated way (even with dummy data) once the system has been implemented. We will perform first the unit testing for each entity, then the integration testing in order to check that the system behaves as expected.\newline\newline
{\large\textbf{Unit testing}}\newline
\textit{The unit testing verifies that each entity of the system works correctly when isolated from the rest of the units.}
\begin{enumerate}
	\item Robot
		\begin{enumerate}
			\item The robot moves and rotates without problems (this can be done manually)
			\item The sonar on the robot is able to detect correctly obstacles
			\item The led on the robot blinks correctly
			\item The robot takes photos
		\end{enumerate}
	\item Console and GUI
		\begin{enumerate}
			\item The distance expression is evaluated correctly
			\item The alarm sound is played
			\item The GUI is able to show the data
		\end{enumerate}
	\item Sonar subsystem
		\begin{enumerate}
			\item Each sonar is able to detect correctly an object in its area
		\end{enumerate}
\end{enumerate}\hfill\break
{\large\textbf{Integration testing}}\newline
\textit{The integration testing verifies that units created and tested independently can coexist and interact among themselves.}
\begin{enumerate}
			\item The console sends a START command to the robot, the robot starts to move.
			\item The console sends a STOP command to the robot, the robot stops itself.
			\item A sonar in the sonar subsystem detects an object, sends the distance value to the console and the consoles evaluates the distance expression correctly.
			\item The console plays a sound alarm if the value calculated by the distance expression is less than the prefixed threshold.
			\item A sonar in the sonar subsystem detects the robot, the robot turns on its left, the led starts to blink, the robot takes a photo and sends it to the console.
			\item After taking the photo, the robot turns on its right to compensate the previous move, the led stops to blink and the robot starts to move again towards the area B.
			\item The robot stops itself when it reaches the area B.
		\end{enumerate}


%===========================================================================
\section{Problem analysis }

%===========================================================================
%===========================================================================
%===========================================================================
%===========================================================================
%===========================================================================
\subsection{Logic architecture}
---TO DO

\subsection{Abstraction gap}
---TO DO

\subsection{Risk analysis}
---TO DO

\section{Hardware}
---TO DO
%===========================================================================
\section{Work Plan}
---TO DO

\section{Project}
---TO DO

\section{Implementation}
---TO DO
\section{Testing}
---TO DO
\section{Deployment}
---TO DO

\section{Maintenance}
---TO DO

\section{About authors}
---TO DO


%===========================================================================
%===========================================================================
\end{document}
 
